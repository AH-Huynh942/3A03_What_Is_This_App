\section{Non-Functional Requirements}
\label{sec:non-functional_requirements}


% Begin Section
\subsection{Look and Feel Requirements}
\label{sub:look_and_feel_requirements}
% Begin SubSection


\subsubsection{Appearance Requirements}
\label{ssub:appearance_requirements}
% Begin SubSubSection
\begin{enumerate}[{LF-A}1. ]
	\item The application shall have a simple design.
	\\ \textbf{Rationale}: A simple design helps keep users from getting overwhelmed. It will also help keep users focused on the 
	functions of the application rather than the design.
	\item The application should have text that is easily legible.
	\\ \textbf{Rationale}: Legible text contributes to ease of use and the opposite may frustrate users.
	\item The application shall have large input boxes and text boxes.
	\\ \textbf{Rationale}: There is a wide array of inputs the user may add. This allows for the user to make sure his/her inputs are correct and accurate.
	\item The application shall have a simple colour palette that prevents colour-blind inaccessibility.
	\\ \textbf{Rationale}: Not having accessibility aspects for the application will prevent a minority of the population from utilizing and testing the app.  
\end{enumerate}
% End SubSubSection


\subsubsection{Style Requirements}
\label{ssub:style_requirements}
% Begin SubSubSection
\begin{enumerate}[{LF-S}1. ]
	\item The application’s components should be evenly spaced.
	\\ \textbf{Rationale}: This allows for navigation to be quick and easy. The user is less prone to mistakes when he/she applies gestures for navigation.
\end{enumerate}
% End SubSubSection

% End SubSection


\subsection{Usability and Humanity Requirements}
\label{sub:usability_and_humanity_requirements}
% Begin SubSection

\subsubsection{Ease of Use Requirements}
\label{ssub:ease_of_use_requirements}
% Begin SubSubSection
\begin{enumerate}[{UH-EOU}1. ]
	\item The system must allow users to report bugs within the application.\\ \textbf{Rationale}: With these reports about bugs within the application, the developers can then investigate and fix the bugs to further improve the application.
	\item The system must allow users to send feedback about the application within the application.\\ \textbf{Rationale}: With feedback received from the users about the application, the developers can then utilize that feedback and further improve the application.
	\item The system must only prompt the user when decisions and confirmations are required.\\ \textbf{Rationale}: If the user is constantly getting prompted unnecessarily, the user will become more annoyed with the application and will have a higher chance in leaving the application.
\end{enumerate}
% End SubSubSection


\subsubsection{Personalization and Internationalization Requirements}
\label{ssub:personalization_and_internationalization_requirements}
% Begin SubSubSection
\begin{enumerate}[{UH-PI}1. ]
	\item The system shall allow the user to select the preferred language used within the application.
	\\ \textbf{Rationale}: Users would then be able to navigate and have a better overall experience with the application when they are able to understand it more easily. Thus, this would then encourage the user to continue using the application in the future.
	\item The system shall allow the user to customize what phrases of the identified language are given to them.
	\\ \textbf{Rationale}: Users would then find more enjoyment while using the application when they can tailor what they want to learn how to say in the language being identified.
\end{enumerate}
% End SubSubSection


\subsubsection{Learning Requirements}
\label{ssub:learning_requirements}
% Begin SubSubSection
\begin{enumerate}[{UH-L}1. ]
	\item The user should be able to figure out how to use the application for the first time without any instruction within 5 minutes.
	\\ \textbf{Rationale}: If the user must spend more then 5 minutes trying to figure out the application or must read a long instruction manual to figure out how to use the application, the user will then become frustrated and abandon the application. This would then discourage the user for ever using the application in the future when they would be frustrated every time they try to use the application.	
\end{enumerate}
% End SubSubSection

\subsubsection{Understandability and Politeness Requirements}
\label{ssub:understandability_and_politeness_requirements}
% Begin SubSubSection
\begin{enumerate}[{UH-UP}1. ]
	\item The user should be able to understand how to use the application smoothly and get a result within 5 minutes of operation.\\ \textbf{Rationale}: If the application is self-intuitive, easy to use, and fast, it would encourage the user to keep using the application and continue using it in the future due to these benefits.
	\item The system shall use proper manners and etiquette whenever the user is prompted to decide or input something.\\ \textbf{Rationale}: If the user feels as if they are being offended or being rude towards, the user would then abandon the application and spread a bad reputation towards the application to other users.
\end{enumerate}
% End SubSubSection


\subsubsection{Accessibility Requirements}
\label{ssub:accessibility_requirements}
% Begin SubSubSection
\begin{enumerate}[{UH-A}1. ]
	\item The system should be able to switch colour settings between a normal colour palette to other colour palettes such as the Protan and Deutan colour palettes \cite{EnChroma2025}, to accommodate people with colour blindness.
	\\ \textbf{Rationale}: Users who are colourblind will then be able to differentiate between different parts of the application, but also be able to use the functionality of the application with more ease.
	
\end{enumerate}
% End SubSubSection

% End SubSection

\subsection{Performance Requirements}
\label{sub:performance_requirements}
% Begin SubSection

\subsubsection{Speed and Latency Requirements}
\label{ssub:speed_and_latency_requirements}
% Begin SubSubSection
\begin{enumerate}[{PR-SL}1. ]
	\item The response time of all the \textbf{API}s used in the app should be less than 1 second.
	\\ \textbf{Rationale}: \textbf{API}s having a response time between 0.1 seconds and 1 second will improve the user's experience and make it have a more competitive edge to other competitors. Having a response time out of this range will casue the users to face minor delays (1 second to 2 seconds), which may lead to them being annoyed by this and swapping to other applications instead.
	\item The time it takes to identify the language cannot take longer than 5 seconds.
	\\ \textbf{Rationale}: The time it takes to identify the language is not as critical as the response time of the \textbf{API}s, since the \textbf{experts} need to analyze the input from the user. However, the wait time for the result must not be too long when it would affect the user's experience of the app.
	\item The app startup time should be less than 2 seconds.
	\\ \textbf{Rationale}: If the startup time takes too long, it would affect the user's experience. This would then make the user more likely to switch to another app when they would get annoyed by it.
\end{enumerate}
% End SubSubSection


\subsubsection{Safety-Critical Requirements}
\label{ssub:safety_critical_requirements}
% Begin SubSubSection
% \begin{enumerate}[{PR-SC}1. ]
	\textbf{N/A}
% \end{enumerate}
% End SubSubSection

\subsubsection{Precision or Accuracy Requirements}
\label{ssub:precision_or_accuracy_requirements}
% Begin SubSubSection
\begin{enumerate}[{PR-PA}1. ]
	\item The app must show the most precise result of identification.
	\\ \textbf{Rationale}: The accuracy of the result from the system is dependent on the quality of the input. The system will sometimes make errors when there is a lack of information.
	\item The average rating of the app must be accurate to one decimal place.
	\\ \textbf{Rationale}: To keep the UI simple for rating, our app would only allow users to submit a rating with a max of one decimal place. This is followed by other apps in the industry. 
\end{enumerate}
% End SubSubSection

\subsubsection{Reliability and Availability Requirements}
\label{ssub:reliability_and_availability_requirements}
% Begin SubSubSection
\begin{enumerate}[{PR-RA}1. ]
	\item The system must be available for 99.9\% of the time, excluding maintenance time and internet network outages.
	\\ \textbf{Rationale}: The longer availability will make the app more competitive. It should have an availability of 99.9\% (three nines) as most of the applications in the industry aim to achieve this \cite{Britannica2025_WritingSystems}. That means the app will only have a downtime of 8.76 hours per year for maintenance and other exceptions. This allows time for adding new \textbf{experts} within the system and solving unexpected crashs/bugs within the system.
	\item The system must create backups of all the app data.
	\\ \textbf{Rationale}: If something unexpected happens, the system could lose all the feedback and information gained from the users. Having a backup of this data will ensure that we are able to recover our lost data if this happens.
\end{enumerate}
% End SubSubSection


\subsubsection{Robustness or Fault-Tolerance Requirements}
\label{ssub:robustness_or_fault_tolerance_requirements}
% Begin SubSubSection
\begin{enumerate}[{PR-RFT}1. ]
	\item The user must be able to enter input that the system could parse (ex. PDF, txt, text) even when the app is not connected to the internet. However, the result provided by the system will not return to the user until the internet connection is restored.
	\\ \textbf{Rationale}: Even though the user cannot access the internet, they should still be able to enter their input as they would not have to enter them again once the internet connection is restored.
	\item The system must be able to handle exceptional input made by the user.
	\\ \textbf{Rationale}: If the input is exceptional, then the app should not crash and instead fail gracefully.
\end{enumerate}
% End SubSubSection

\subsubsection{Capacity Requirements}
\label{ssub:capacity_requirements}
% Begin SubSubSection
\begin{enumerate}[{PR-C}1. ]
	\item The system must be able to handle 10 simultaneous requests.
	\\ \textbf{Rationale}: Considering the scope of the project, this requirement can be verified without taking much time.	
\end{enumerate}
% End SubSubSection

\subsubsection{Scalability or Extensibility Requirements}
\label{ssub:scalability_or_extensibility_requirements}
% Begin SubSubSection
\begin{enumerate}[{PR-SE}1. ]
	\item The code for the system must be open for addition, but closed for modification.
	\\ \textbf{Rationale}: By applying this principle, it ensures stability and consistency (close to modification) and maximizes the scalability and extensibility as well.
\end{enumerate}
% End SubSubSection

\subsubsection{Longevity Requirements}
\label{ssub:longevity_requirements}
% Begin SubSubSection
\textbf{N/A}
% \begin{enumerate}[{PR-L}1. ]
% 	\item 
% \end{enumerate}
% End SubSubSection

% End SubSection

\subsection{Operational and Environmental Requirements}
\label{sub:operational_and_environmental_requirements}
% Begin SubSection

\subsubsection{Expected Physical Environment}
\label{ssub:expected_physical_environment}
% Begin SubSubSection
\begin{enumerate}[{OE-EPE}1. ]
	\item The system must be compatible with ARM architecture paired with both Android and iOS operating systems.
	\\ \textbf{Rationale}: 99\% of smartphones use ARM architecture \cite{ARM2025}, with the listed operating systems holding over 99\% of the market share for mobile devices worldwide \cite{Backlinko2025}.
\end{enumerate}
% End SubSubSection

\subsubsection{Requirements for Interfacing with Adjacent Systems}
\label{ssub:requirements_for_interfacing_with_adjacent_systems}
% Begin SubSubSection
\begin{enumerate}[{OE-IA}1.]
	\item The system must perform two-way communication with a cloud database in at most 1 second.
	\textbf{Rationale}: Each expert must interact with the central database to reference their respective data stores, and obtain necessary language information. A max latency of 1 second ensures fast communication between both parties so that the language is deduced in a sensible time.   
	\item The system must be able to send JPEG, HEIC, and PNG images of up to 100MB to a computer vision \textbf{API}, and receive information without disruption.
	\\ \textbf{Rationale}: Phones that use iOS (iPhone) store images in HEIC or JPEG by default. Phones that use Android operating systems most commonly store them in JPEG and PNG \cite{Android2025}. Image sizes of up to 100MB ensure high quality pictures are transferable between the system and \textbf{API}.
	\item The system must be able to dynamically send country names and receive a corresponding photo from an image generator \textbf{API}.
	\\ \textbf{Rationale}: The innovative feature requires that images must be able to dynamically produce a map highlighting countries with prominent speakers for certain languages.
\end{enumerate}
% End SubSubSection

\subsubsection{Productization Requirements}
\label{ssub:productization_requirements}
% Begin SubSubSection
\textbf{N/A}
% \begin{enumerate}[{OE-P}1. ]
% 	\item 
% \end{enumerate}
% End SubSubSection

\subsubsection{Release Requirements}
\label{ssub:release_requirements}
% Begin SubSubSection
\begin{enumerate}[{OE-R}1. ]
	\item The system must be compatible with Android 12.0, iOS 18.0, or above.
	\\ \textbf{Rationale}: Both iOS and Android operating systems are used by the app, and these are their latest versions that receive active maintenance and security support \cite{EndOfLife2025_Android,EndOfLife2025_iOS}. 
	\item The deployed system must have a maximum downtime of 1 hour per 2 weeks.
	\\ \textbf{Rationale}: The product must be available to use for as long as possible, and 1 hour per week leaves sufficient time to deploy maintenance updates and bug fixing.
	\item The system must only be accessible for users with a network connection of at least 2 Mb/s.
	\\ \textbf{Rationale}: To guarantee all \textbf{API} and database communication functions as expected, a stable network connection is necessary. These calls apply a load comparable to those made when browsing the web, where a speed of 2-5 Mb/s is satisfactory \cite{Speedtest2025}.
\end{enumerate}
% End SubSubSection

% End SubSection

\subsection{Maintainability and Support Requirements}
\label{sub:maintainability_and_support_requirements}
% Begin SubSection

\subsubsection{Maintenance Requirements}
\label{ssub:maintenance_requirements}
% Begin SubSubSection
\begin{enumerate}[{MS-M}1. ]
	\item The system must have a minimum number of monthly software updates to patch bugs in the software.  
	\\ \textbf{Rationale}: This ensures the system runs smoothly and that bugs gets patched and fixed regularly. This will help to maintain 	good app quality throughout its use.
	\item The system must have a minimum of three times per year to expand and update language libraries. 
	\\ \textbf{Rationale}: This ensures the new created slangs and words are added into the library to help identify the language.
\end{enumerate}
% End SubSubSection

\subsubsection{Supportability Requirements}
\label{ssub:supportability_requirements}
% Begin SubSubSection
\begin{enumerate}[{MS-S}1. ]
	\item The system must be able to support different operating systems such as iOS and Android. 
	\\ \textbf{Rationale}: This ensures that the large variety of phone users are able to use the app. 
\end{enumerate}
% End SubSubSection

\subsubsection{Adaptability Requirements}
\label{ssub:adaptability_requirements}
% Begin SubSubSection
\begin{enumerate}[{MS-A}1. ]
	\item The system shall be able to run on the most current Android release on Android devices. 
	\\ \textbf{Rationale}: This ensures that the app is compatible with older and new Android devices.
\end{enumerate}
% End SubSubSection

% End SubSection

\subsection{Security Requirements}
\label{sub:security_requirements}
% Begin SubSection

\subsubsection{Access Requirements}
\label{ssub:access_requirements}
% Begin SubSubSection
\begin{enumerate}[{SR-AC}1. ]
	\item The app must ask the user for file storage, paste board, and photos of the device to acquire the input. 
	\\ \textbf{Rationale}: The system must ask the permission of accessing format that stores text in order to identify language. 
\end{enumerate}
% End SubSubSection

\subsubsection{Integrity Requirements}
\label{ssub:integrity_requirements}
% Begin SubSubSection
\textbf{N/A}
% \begin{enumerate}[{SR-INT}1. ]
% 	\item 
% 	\textbf{Rationale}:
% \end{enumerate}
% End SubSubSection

\subsubsection{Privacy Requirements}
\label{ssub:privacy_requirements}
% Begin SubSubSection
\begin{enumerate}[{SR-P}1. ]
	\item The app must ensure the confidentiality of user input. 
	\\ \textbf{Rationale}: Input from the user used to identify the language should not be leaked to anyone.
\end{enumerate}
% End SubSubSection

\subsubsection{Audit Requirements}
\label{ssub:audit_requirements}
% Begin SubSubSection
\textbf{N/A}
% \begin{enumerate}[{SR-AU}1. ]
% 	\item 
% 	\textbf{Rationale}:
% \end{enumerate}
% End SubSubSection

\subsubsection{Immunity Requirements}
\label{ssub:immunity_requirements}
% Begin SubSubSection
\begin{enumerate}[{SR-IM}1. ]
	\item The system must not accept unexpected input.
	\\ \textbf{Rationale}: This is to prevent attacks like SQL Injection.
\end{enumerate}
% End SubSubSection

% End SubSection

\subsection{Cultural and Political Requirements}
\label{sub:cultural_and_political_requirements}
% Begin SubSection

\subsubsection{Cultural Requirements}
\label{ssub:cultural_requirements}
% Begin SubSubSection
\begin{enumerate}[{CP-C}1. ]
	\item The system must not contain any words that can be considered offensive ( to any religion, ethnicity, disability, sexual orientation, groups) where the app is available, utilizing a list of unacceptable words for reference.
	\\ \textbf{Rationale}: This is to ensure people feel safe when using the app.  Any words well known to be offensive must purposely be 	excluded from any designs or usage in the system. 
\end{enumerate}
% End SubSubSection

\subsubsection{Political Requirements}
\label{ssub:political_requirements}
% Begin SubSubSection
\textbf{N/A}
% \begin{enumerate}[{CP-P}1. ]
% 	\item 
% 	\textbf{Rationale}:
% \end{enumerate}
% End SubSubSection

% End SubSection

\subsection{Legal Requirements}
\label{sub:legal_requirements}
% Begin SubSection

\subsubsection{Compliance Requirements}
\label{ssub:compliance_requirements}
% Begin SubSubSection
\begin{enumerate}[{LR-COMP}1. ]
	\item The system must convey that all images passed into the system are to be analyzed by computer vision algorithms prior to any collection.
	\\ \textbf{Rationale}: Based on \textbf{PIPEDA} Principle 2, the purpose of collected data must be specified by the organization before or during the time of collection \cite{PIPEDA2025}. All apps launched in Canada must comply with \textbf{PIPEDA} regulations.
	\item All images processed with computer vision algorithms must ensure that only text is extracted and analyzed.
	\\ \textbf{Rationale}: Based on \textbf{PIPEDA} Principle 4, only data that is necessary for the system is collected \cite{PIPEDA2025}. So, any other visible artifacts including (but not limited to) addresses, signatures, and people cannot be analyzed.
	\item Collected images are discarded upon analysis of the text. The text is also discarded when the user prompts to close the window.
	\\ \textbf{Rationale}: Based on \textbf{ISO} 27001, data must be encrypted when it is being transmitted between parties to prevent leakage of any IP addresses and other personal data \cite{ISMS2025}.
\end{enumerate}
% End SubSubSection

\subsubsection{Standards Requirements}
\label{ssub:standards_requirements}
% Begin SubSubSection
\begin{enumerate}[{LR-STD}1. ]
	\item The system must ensure that text is resizable, and that the interface’s colour contrast satisfies the Web Content Accessibility Guidelines.
	\\ \textbf{Rationale}: This is a key standard outlined in the \textbf{AODA}, which applies to \textbf{Languify} as it is developed and released in Ontario \cite{Speedtest2025}.
	\item The system shall offer a virtual keyboard for each supported script, which is controllable with the physical keyboard’s arrow keys. Each screen must be fully navigable with the physical keyboard as well.
	\\ \textbf{Rationale}: As per the \textbf{AODA}, systems must be navigable without a mouse to be considered operable \cite{Speedtest2025}. This also ensures it is usable under any conditions.
	\item The system shall provide clearly labelled buttons, input modules, and instructions for every functionality.
	\\ \textbf{Rationale}: As per the \textbf{AODA}, systems must be understandable, with unambiguous wording and input assistance to make them as easy to use as possible \cite{AODA2025}.
	\item The system must be compatible with screen readers and provide alternative text when needed.
	\\ \textbf{Rationale}: As per the \textbf{AODA}, it is crucial to follow proper coding standards and semantics so that they are accessible to as many audiences as possible \cite{AODA2025}.
\end{enumerate}
% End SubSubSection

% End SubSection

% End Section