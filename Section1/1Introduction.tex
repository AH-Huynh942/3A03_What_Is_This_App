\section{Introduction}
\label{sec:introduction}
% Begin Section
dfsafdkj;slafhdks;..;ahf;sdklahf;ldsa

\subsection{Purpose}
\label{sub:purpose}
% Begin SubSection
blah...ahdsafdaslfhdsklahdsafdaslfhdsklahdsafdaslfhdsklahdsafdaslfhdsklahdsafdaslfhdsklahdsafdaslfhdsklahdsafdaslfhdskl
% End SubSection


\subsection{System Description}
\label{sub:system_description}
% Begin SubSection
The overall system follows a Pipe and Filter architecture, with three subsystems, each having its own architectural style. The Identifier subsystem is structured using a Blackboard architecture because it contains passive agents that communicate with an active data store. The Reader subsystem follows a Pipe and Filter architecture since it processes the user input step by step to ensure compatibility with the Identifier subsystem. The Display subsystem follows a Repository architecture, as it retrieves data from a data store that supplements language detection by providing interesting facts related to the identified language.
% End SubSection


\subsection{Overview}
\label{sub:overview}
% Begin SubSection
blah...ahdsafdaslfhdsblah...ahdsafdaslfhdsblah...ahdsafdaslfhdsblah...ahdsafdaslfhds
% End SubSection


% End Section